\documentclass{article}
\usepackage{fancyhdr}
\usepackage{titlesec}
\usepackage{graphicx}
\graphicspath{ {./img/} }

\pagestyle{fancy}
\fancyhf{}
\lhead{Modul 6 Praktikum Jaringan Komputer}
\rfoot{\footnotesize Page \thepage}
\lfoot{\footnotesize Mahyus Ihsan, S.Si, M.Si \newline Jurusan Informatika Universitas Syiah Kuala \newline Modul oleh : Diky Wahyudi, Furqan Al Ghifari, Rendika Rahmaturrizki}
\renewcommand{\headrulewidth}{1pt}
\renewcommand{\footrulewidth}{1pt}

\titleformat*{\section}{\small\bfseries}

\begin{document}
    \begin{center}
        \textbf{Modul 6 Praktikum Jaringan Komputer}

        \textbf{ICMP dan Basic Router Configuration}
    \end{center}

    \section*{Deskripsi Singkat}
    \begin{flushleft}
        Internet Control Message Protocol (ICMP) adalah  jaringan protokol yang bertanggung jawab untuk melaporkan kesalahan melalui cara menghasilkan dan mengirim pesan ke alamat IP sumber ketika ada masalah jaringan pada sistem. 
    \end{flushleft}

    \section*{Tujuan}
    \begin{enumerate}
        \item Dapat memahami tentang ICMP dan cara kerjanya
        \item Dapat melakukan konfigurasi dasar pada sebuah router
    \end{enumerate}

    \begin{flushleft}
        \textbf{Materi 1 - ICMP}
        \newline

        Isi materi 1
    \end{flushleft}

    \begin{flushleft}
        \textbf{Materi 2 - NamaMateri}
        \newline

        Isi materi 2
    \end{flushleft}

    \begin{flushleft}
        \textbf{Materi 3 - NamaMateri}
        \newline

        Isi materi 3
    \end{flushleft}

    \newpage
    \begin{flushleft}
        \textbf{Tugas}
        \newline

        \begin{enumerate}
            \item Tugas 1
            \item Tugas 2
        \end{enumerate}
    \end{flushleft}
\end{document}
