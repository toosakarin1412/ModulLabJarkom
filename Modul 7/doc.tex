\documentclass{article}
\usepackage{fancyhdr}
\usepackage{titlesec}
\usepackage{graphicx}
\graphicspath{ {./img/} }

\pagestyle{fancy}
\fancyhf{}
\lhead{Modul 7 Praktikum Jaringan Komputer}
\rfoot{\footnotesize Page \thepage}
\lfoot{\footnotesize Mahyus Ihsan, S.Si, M.Si \newline Jurusan Informatika Universitas Syiah Kuala \newline Modul oleh : Diky Wahyudi, Furqan Al Ghifari, Rendika Rahmaturrizki}
\renewcommand{\headrulewidth}{1pt}
\renewcommand{\footrulewidth}{1pt}

\titleformat*{\section}{\small\bfseries}

\begin{document}
    \begin{center}
        \textbf{Modul 7 Praktikum Jaringan Komputer}

        \textbf{Transport Layer dan Aplication Layer}
    \end{center}

    \section*{Deskripsi Singkat}

    \begin{flushleft}
        \textbf{Transport layer} merupakan sebuah lapisan transportasi. 
        Transport layer ini dapat menggabungkan beberapa koneksi transport ke dalam jaringan koneksi yang sama. 
        Transport Layer bertanggung jawab untuk menyampaikan data ke proses aplikasi yang sesuai pada komputer host.
        \newline

        \textbf{Application layer} merupakan layer ketujuh yang ada dalam Open Systems Interconnection (OSI) model dan menjadi satu-satunya layer yang dapat secara langsung berinteraksi dengan end user. 
        Layer ini terletak pada tingkat paling atas dan diizinkan oleh perangkat lunak atau user untuk mendapatkan akses ke jaringan. 
    \end{flushleft}

    \section*{Tujuan}
    \begin{enumerate}
        \item Dapat memahami konsep penggunaan port
        \item Dapat memahami berbagai macam protokol yang ada pada Aplication Layer
        \item Dapat melakukan konfigurasi DHCP server dan DNS server
    \end{enumerate}

    \begin{flushleft}
        \textbf{Materi 1 - Port pada transport layer}
        \newline

        Isi materi 1
    \end{flushleft}

    \begin{flushleft}
        \textbf{Materi 2 - Protokol pada Aplication Layer}
        \newline

        Isi materi 2
    \end{flushleft}

    \begin{flushleft}
        \textbf{Materi 3 - Konfigurasi DHCP server dan DNS server}
        \newline

        Isi materi 3
    \end{flushleft}

    \newpage
    \begin{flushleft}
        \textbf{Tugas}
        \newline

        \begin{enumerate}
            \item Tugas 1
            \item Tugas 2
        \end{enumerate}
    \end{flushleft}
\end{document}
