\documentclass{article}
\usepackage{fancyhdr}
\usepackage{titlesec}
\usepackage{graphicx}
\graphicspath{ {./img/} }
\usepackage{multirow}
\usepackage[a4paper, total={6in, 8in}]{geometry} % use this package in all paper
\usepackage{hyperref} 

\pagestyle{fancy}
\fancyhf{}
\lhead{Modul 11 Praktikum Jaringan Komputer}
\rfoot{\footnotesize Page \thepage}
\lfoot{\footnotesize Mahyus Ihsan, S.Si, M.Si \newline Jurusan Informatika Universitas Syiah Kuala \newline Modul oleh : Diky Wahyudi, Furqan Al Ghifari, Rendika Rahmaturrizki}
\renewcommand{\headrulewidth}{1pt}
\renewcommand{\footrulewidth}{1pt}

\titleformat*{\section}{\small\bfseries}

\begin{document}
    \begin{center}
        \textbf{Modul 11 Praktikum Jaringan Komputer}

        \textbf{MikroTik}
    \end{center}

    \section*{Deskripsi Singkat}
    \hspace{\parindent} Mikrotik merupakan sistem operasi berupa perangkat lunak yang 
    digunakan untuk menjadikan komputer menjadi router jaringan. 
    Sistem operasi ini sangat cocok untuk keperluan administrasi 
    jaringan komputer, misalnya untuk membangun sistem jaringan komputer 
    skala kecil maupun besar.\\

    MikroTik RouterOS dapat di install pada komputer maupun laptop serta 
    dapat juga diinstall di virtual machine seperti VirtualBox. 
    Untuk menginstall MikroTik RouterOS diperlukan file ISO yang dapat diperoleh 
    melaui situs mikrotik.com. Sistem operasi MikroTik ini hanya bersifat 
    trial dengan durasi waktu 24 jam. Untuk versi penuhnya (full version) 
    dapat dilakukan dengan cara membeli lisensi dari penyedia MikroTik RouterOS. 
    Pada kesempatan kali ini, kita akan mencoba menyajikan panduan instalasi 
    MikroTik di VirtualBox.


    \section*{Tujuan}
    \begin{enumerate}
        \item Mengetahui fungsi-fungsi MikroTik 
        \item Mampu melakukan instalasi MikroTik Router OS di virtualbox 
        \item Mampu melakukan konfigurasi dasar MikroTik di virtualbox
        \newline 
    \end{enumerate}


    % theory section

    \begin{flushleft}
        \textbf{Materi 1 - Instalasi Mikrotik RouterOS dan Windows XP di VirtualBox}

        \begin{center}
            \footnotesize\textbf{* Note :} Windows XP yang diinstall akan digunakan 
            sebagai host di dalam jaringan.\\
        \end{center}

        Link Download file iso : 
        \href{https://www.microsoft.com/en-us/Download/confirmation.aspx?id=18242}{Windows XP} \;
        \href{https://mikrotik.com/download/archive}{Router Mikrotik OS} \\

        Cara mendownload MikroTik OS dari website :
        \href{https://www.youtube.com/watch?v=OMsGPKRLabM}{cara mendownload mikrotik os}\\

        Langkah melakukan instalasi Mikrotik dan Windows XP di virtualbox :
        \href{https://www.youtube.com/watch?v=LlmNKEuiWvk&list=PLvT7-AKYOYMxXuwYv1uA9e0aDsX1XgF4l&index=2}{cara menginstall windows xp dan mikrotik os}\\

    \end{flushleft}

    \begin{flushleft}
        \textbf{Materi 2 - Konfigurasi MikroTik pada VirtualBox}
        \newline

        Cara melakukan konfigurasi dasar MikroTik (part I) : \href{https://www.youtube.com/watch?v=qNhfV5cta50&list=PLvT7-AKYOYMxXuwYv1uA9e0aDsX1XgF4l}{konfigurasi mikrotik dasar}\\
        Cara melakukan konfigurasi dasar MikroTik (part II) : \href{https://www.youtube.com/watch?v=3yU42r7QvSU&list=PLvT7-AKYOYMxXuwYv1uA9e0aDsX1XgF4l&index=3}{konfigurasi mikrotik lanjutan}\\

    \end{flushleft}

    \begin{flushleft}
        \textbf{Bonus - Mempelajari fitur fitur yang dapat dilakukan menggunakan MikroTik}
        \newline

        Untuk materi lebih lanjut terkait dengan pembelajaran MikroTik, silahkan kunjungi website 
        \href{https://www.youtube.com/playlist?list=PLvT7-AKYOYMxXuwYv1uA9e0aDsX1XgF4l}{Seri Belajar Mikrotik}\\

        \begin{center}
            \footnotesize\textbf{* Note :} Jika anda belum memiliki device MikroTik, anda dapat 
            menggunakan VirtualBox sebagai sarana pembelajaran simulasi\\
        \end{center}

    \end{flushleft}


    % practice section
%     \newpage
%     \begin{flushleft}
%         \textbf{Tugas}
%         \newline

%         \begin{enumerate}
%             \item Tugas 1
%             \item Tugas 2
%         \end{enumerate}
%     \end{flushleft}

 \end{document}

